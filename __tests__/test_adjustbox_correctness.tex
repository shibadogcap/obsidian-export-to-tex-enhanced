\documentclass[paper=a4]{jlreq}
\usepackage{amsmath}
\usepackage{amssymb}
\usepackage{amsthm}
\usepackage{amsfonts}
\usepackage{mathtools}
\usepackage{graphicx}
\usepackage{multirow}
\usepackage{hyperref}
\usepackage{diffcoeff}
\usepackage{comment}
\usepackage{mhchem}
\usepackage[separate-uncertainty]{siunitx}
\usepackage{newunicodechar}
\usepackage{listings}
\usepackage{float}
\usepackage{lscape}
\usepackage{adjustbox}
\usepackage{tabularx}
\usepackage{booktabs}

\begin{document}

% テスト1: 通常テーブル(2列)
\begin{table}[h]
\centering
\begin{adjustbox}{max width=\textwidth}
\begin{tabularx}{\textwidth}{XX}
\toprule
Header1 & Header2 \\
\midrule
Data1 & Data2 \\
Data3 & Data4 \\
\bottomrule
\end{tabularx}
\end{adjustbox}
\caption{テスト1: 通常テーブル}
\end{table}

% テスト2: 中列テーブル(7列)
\begin{table}[h]
\centering
{\small
\begin{adjustbox}{max width=\textwidth, max height=0.85\textheight}
\begin{tabularx}{\textwidth}{XXXXXXX}
\toprule
H1 & H2 & H3 & H4 & H5 & H6 & H7 \\
\midrule
D1 & D2 & D3 & D4 & D5 & D6 & D7 \\
D8 & D9 & D10 & D11 & D12 & D13 & D14 \\
\bottomrule
\end{tabularx}
\end{adjustbox}
}
\caption{テスト2: 中列テーブル(7列)}
\end{table}

% テスト3: 広いテーブル(12列)landscape
\begin{landscape}
\begin{table}[h]
\centering
\begin{adjustbox}{max width=\textwidth}
\begin{tabularx}{\textwidth}{XXXXXXXXXXXX}
\toprule
H1 & H2 & H3 & H4 & H5 & H6 & H7 & H8 & H9 & H10 & H11 & H12 \\
\midrule
D1 & D2 & D3 & D4 & D5 & D6 & D7 & D8 & D9 & D10 & D11 & D12 \\
D13 & D14 & D15 & D16 & D17 & D18 & D19 & D20 & D21 & D22 & D23 & D24 \\
\bottomrule
\end{tabularx}
\end{adjustbox}
\caption{テスト3: 広いテーブル(12列)}
\end{table}
\end{landscape}

\end{document}
