\documentclass[paper=a4]{jlreq}
\usepackage{amsmath}
\usepackage{amssymb}
\usepackage{amsthm}
\usepackage{amsfonts}
\usepackage{mathtools}
\usepackage{graphicx}
\usepackage{multirow}
\usepackage{hyperref}
\usepackage{diffcoeff}
\usepackage{comment}
\usepackage{mhchem}
\usepackage[separate-uncertainty]{siunitx}
\usepackage[math-style=ISO,warnings-off={mathtools-colon,mathtools-overbracket}]{unicode-math}
\usepackage{newunicodechar}
\usepackage{listings}
\usepackage{float}
\usepackage{adjustbox}
\usepackage{tabularx}
\usepackage{booktabs}
%%%%%%%%%%%%%%%%%%%%%%%%%%%%%%%%%%%%%%%%%%%%%%%%%
\newunicodechar{、}{,}
\newunicodechar{。}{.}
\NewDocumentCommand\、{}{{\char"3001}}
\NewDocumentCommand\。{}{{\char"3002}}
\NewDocumentCommand\degC{}{\ensuremath{^\circ\symup{C}}}
\NewDocumentCommand\abs{m}{\left|#1\right|}
%%%%%%%%%%%%%%%%%%%%%%%%%%%%%%%%%%%%%%%%%%%%%%%%%
\jlreqsetup{
    appendix_counter={
        section={
            value=0,
            the={\Alph{section}}
        },
        table={
            value=0,
            the={\Alph{section}\arabic{table}}
        },
        figure={
            value=0,
            the={\Alph{section}\arabic{figure}}
        }
    },
    appendix_heading={
        section={
            label_format={付録\thesection:}
        }
    }
}

\title{テスト: 画像とテーブルレイアウト改善}
\author{テストユーザー}
\date{2025-11-01}

\begin{document}
\maketitle

\section{テスト内容}

\subsection{画像テスト}

以下のような画像を含むテストです。

\begin{figure}[h]
\centering
\includegraphics[width=0.8\textwidth,keepaspectratio]{放射線_実験装置.png}
\caption{実験装置(改善版)}
\end{figure}

画像は自動的に \texttt{0.8\textbackslash textwidth} に制限されます。

\subsection{テーブルテスト}

以下のテーブルをテストします。

\begin{table}[h]
\centering
\begin{adjustbox}{max width=\textwidth}
\begin{tabularx}{\textwidth}{XXX}
\toprule
項目 & 説明 & 備考 \\
\midrule
画像幅 & テキスト幅の80\% & keepaspectratioで比率保持 \\
テーブル & tabularx使用 & 自動カラム幅調整 \\
線のスタイル & booktabs & プロ品質の表示 \\
\bottomrule
\end{tabularx}
\end{adjustbox}
\caption{改善内容}
\end{table}

\subsection{複雑なテーブルテスト}

より複雑なテーブルの例:

\begin{table}[h]
\centering
\begin{adjustbox}{max width=\textwidth}
\begin{tabularx}{\textwidth}{XXXX}
\toprule
No. & 実験項目 & 計測値 & 結果 \\
\midrule
1 & 自然計数の測定 & 20回測定 & ポアッソン分布に従う \\
2 & ベータ線計数分布 & 100回, 300回計測 & 平均値65付近 \\
3 & 距離依存性 & 30mm, 120mm & 距離で大きく変化 \\
4 & 吸収実験(Ti箔) & 0-6枚 & 線吸収係数μ=0.075 \\
5 & 吸収実験(Cu箔) & 0-6枚 & 線吸収係数μ=0.030 \\
\bottomrule
\end{tabularx}
\end{adjustbox}
\caption{実験結果サマリー}
\end{table}

\subsection{位置指定のテスト}

図表の位置指定オプション:
\begin{itemize}
\item h = ここに配置
\item t = ページ上部
\item b = ページ下部
\item p = 別ページ
\item H = 強制ここ(floatパッケージ必須)
\end{itemize}

デフォルトは \texttt{[h]} 設定になりました。

\end{document}
